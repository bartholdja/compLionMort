\documentclass[12pt,oneside, a4paper,openright, fleqn, titlepage]{article}
\begin{document}
Workers usually omit males from population models of mammals because records on both birth and death times (i.e. life span data) for wild males is sparse. Collecting life span data for males is often harder than for females due to sex differences in dispersal. In many mammal species, females are philopatric while maturing males leave their natal place or social unit . When they disperse, they commonly emigate from the area of capture-recapture/recovery (CRR) field sites . They are subsequently lost for data collection and their ages at death remain unknown (i.e. right-censored records). % we don't actually know that
Even more uncertainty about their ages at death is often introduced by the fact that a missing male may have dispersed or died. In addition to emigrating males, immigrating males further complicate the collection of life span data because their birth dates are unobserved (i.e. left-truncated records). Sex differences in dispersal therefore increase the number of truncated and censored records in CRR/life span data and cause a sex bias among truncated records. CRR data always contain some incomplete records, because some individuals were already alive when the study started or are still alive at the end of it. recently developed a method to infer age-specific survival that accounts for uncertainty in times of birth and death. However, the need persists to extend the method to account for further uncertainty in male ages at death due to male-biased dispersal.
\end{document}