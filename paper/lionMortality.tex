% Abstract
\section*{Abstract}
%
\section*{Key words}
age-specific survival, Bayesian inference, maximum likelihood, dispersal, sex differences in survival, African lions
%
\section*{Introduction}
Researchers commonly omit males from demographic analyses of mammals because records on both birth and death times (i.e. life span data) for wild males is sparse. Collecting life span data for males is often harder than for females due to sex differences in dispersal. In many mammal species, females are philopatric while maturing males leave their natal place or social unit \citep{ Greenwood:1980bp, Handley:2007tu}. When they disperse, they commonly emigrate from the area of capture-recapture/recovery (CRR) field sites \citep[e.g.][]{Pusey:1987vc}. They are subsequently lost for data collection and their ages at death remain unobserved (i.e. right-censored records). Even more uncertainty about their ages at death is often introduced by the fact that a missing male may have dispersed or died. In addition to emigrating males, immigrating males further complicate the collection of life span data because their birth dates are unobserved (i.e. left-truncated records). Sex differences in dispersal therefore increase the number of truncated and censored records in CRR/life span data and cause a sex bias among truncated records. CRR data always contain some incomplete records, because some individuals were already alive when the study started or are still alive at the end of it. \citet{Colchero:2012bf} recently developed a method to infer age-specific survival that accounts for uncertainty in times of birth and death. However, the need persists to extend the method to account for further uncertainty in male ages at death due to male-biased dispersal.

Thorough knowledge about age-specific survival along with age-specific reproduction of both sexes is crucial for studying both ecological and evolutionary processes \citep{Coulson:2010kj, Metcalf:2007tk}. %Reference does only fit for need of age-specific demographic rates, not for "both sexes". 
For example, assuming flat survival across adult ages can result in false estimates of population growth rates (Colchero et al 2014). Therefore, we need to understand how survival changes with age in order to project changes in the size and age structure of populations for management purposes \citep{Pollock:1981tk}. Furthermore, many researcher build female-only population models and assume that the female population captures the development of the total population sufficiently well. However, in most mammal species the population age structure differs between the sexes due to sex differences in sex ratios at birth and age-specific survival \citep{CluttonBrock:2007vl, Promislow:1992uc}. In these cases, quantities calculated from female-only model, such as lifetime reproductive success or life expectancy, cannot simply be extrapolated to the male population. In addition, under circumstances when the age structure of the male population affects the development of the female population, female-only models may result in wrong projections of population development \citep{Whitman:2004gp, Whitman:2007iu}. Researchers rarely challenge the practice of omitting males from demographic models (Barthold et al. 2014) not least because we lack methods to infer the necessary sex- and age-specific survival from incomplete male life span data. The lack of good estimates of male age-specific survival also deters evolutionary studies. Age-specific survival is a key element of any measure of fitness in age structured populations \citep{Metcalf:2007tk}. Without a measure of male fitness, researchers cannot apply the fitness-maximising principle in studies of the evolution of male life histories and the diversity of male life histories across species (REF).

Age-specific survival is now routinely inferred from CRR data \citep{Pollock:2000id}. Field ecologists collect CRR data from populations by catching, marking, and releasing individuals, mostly of unknown age, which are then re-captured, not detected, or recovered dead on subsequent sampling occasions \citep{Catchpole:1998th}. As a result, CRR data contain the capture histories of some individuals from birth to death but also many left-truncated and/or right-censored records \citep{Colchero:2012bf}. Models based on the Cormack--Jolly--Seber framework \citep[CSJ][]{Cormack:1964vi,Jolly:1965wt,Seber:1965tj} can include both uncensored and right-censored records. Generalizations of the basic CJS framework can also accommodate multiple states, such as location and developmental stage \citep{NeilArnason:1973kz,Schwarz:1993va,Lebreton:2002ju}. To include left-truncated records workers either assume that mortality is constant with age \citep{Aebischer:1990vp} (other REF?) or use time at capture as a surrogate for time at birth \citep{Crespin:2006jg,Reed:2008wa}. However, since both approaches can bias survival estimates, most workers omit left-truncated records.  Others have developed other ways to impute unknown ages... . \citet{Colchero:2012bf}  have recently developed an alternative approach that combines estimation of survival parameters and imputation of unknown times of birth and death within a Bayesian hierarchical framework. By modelling both unknown birth and death times as latent variables, combined with a flexible parametric mortality function for the full population, they can admit partial observations on individuals of unknown age, extending the types of observations that can be included to obtain population-level estimates of survival \citet{Colchero:2012bf}. In order to make the method available to researchers and population managers without extensive knowledge in Bayesian statistics and programming, \citep{Colchero:2012db} wrote the package "BaSTA" in the statistical computing language R \citep{Team:2012wf}. To date, BaSTA can contain sex as a state but the framework assumes equal recapture probabilities for both sexes. Since recapture probabilities of emigrating males drops to 0, this is a false assumption, that results in the underestimation of male survival.

To our knowledge, no age-specific survival estimates have been published for male African lions (\emph{Panthero leo}). Two-sex lion population models therefore either use stage-specific survival estimates \citep{Whitman:2004gp} or female age-specific survival \citep{Becker:2013ib,Packer:1998vr} to approximate male survival. However, comparative morphology across sexually dimorphic species indicates that male survival  should be lower than female survival in African lions and tentative analyses support this hypothesis \citep{CluttonBrock:2007vl,  Packer:1988ux, Promislow:1992uc}. Furthermore, age-specific survival estimates, or a combination of age- and stage-specific estimates, may proof better suited to accurately capture lion population dynamics than stage-specific survival alone. The lack of male age-specific survival persists despite a multitude of lion field studies \citep{Packer:2013jm} due to the complex data structure that arises from male-biased dispersal. Lions show the typical pattern of dispersal of polygynous social mammals. Maturing males disperse between the ages of 2 to 4 years, while most females are philopatric and stay within or close to their birth pride \citep{Pusey:1987vc}.  Life span data on males is therefore commonly right-censored for emigrants and left-truncated for immigrants, which prevents inference on male survival using available methods. In addition, estimating early life survival is hindered by the fact that the a large proportion of lions that die before the age of 2 remain unsexed. Previous analyses have therefore not distinguished early survival between the sexes \citep{Whitman:2004gp}. Secondary dispersal... 

Here, we adapt and extend the Bayesian hierarchical framework for survival inferences on left-truncated and right-censored CRR data of \citet{Colchero:2012bf} to provide age-specific survival estimates for both sexes of African lions. The original framework estimates survival by using a parametric mortality model and imputing unknown stages of birth and death times. The model is in discrete time steps that represents the sampling intervals. It takes sex as an additional state and allows the estimation of sex-specific survival but assumes equal re-sighting probabilities for males and females. We adapted the model to accommodate the data structure that arises from male-biased dispersal in general and to the specific structure of lion data sets from observational field studies. We used data from lions at Hwange National Park (HNP), Zimbabwe. We inferred age-specific survival from age at emergence from birth dens (X months of age), because cubs are not seen before that. Since individuals without an observed birth date get an expert judged estimated birth date, we only needed to accommodate right-censoring. We included sex-specific resighting probabilities that vary with age. We imputed sex as an unknown state for unsexed individuals. We changed the model to continuous time because in the lion data the individuals histories are collapsed into first seen and last seen dates (i.e. life span data), which rendered discrete time steps to represent sampling intervals unnecessary. We tested different mortality models since Hwange lions experience a high degree of human-induced external mortality and this can be reflected differently well by different parametric mortality models (REFS). We simulated a data set using known mortality and dispersal rate parameters and tested our model by checking if we could retrieve the mortality parameters.  We also used this simulated data to test how sensitive the model is towards misspecification of male age-specific emigration probabilities, because most studies don't have the data to estimate these and will therefore enter estimated guesses into the model. We predicted that male survival would be lower than female survival and that males suffered a higher degree of age-independent external mortality (reflected by a constant parameter in the survival model), and also aged faster (internal and age-dependent external mortality). We found that the bathtub shaped Gompertz-Makeham model was the best model for mortality in Hwange for both sexes. We found that males lived shorter, aged faster, and had a higher degree of age-independent mortality, indicated by a higher Makeham term, than females. Most importantly, we found that our model could retrieve the mortality parameter from the simulated data set, and that while the model is sensitive to gross misspecification of male resighting probability, it performed well using an estimated derived from the data. 
% Difference between CRR data and life span data, CRR data that are post-hoc, after an individual disappeared or died, collapsed into a first seen and last seen date. Common practice for observational field studies of large mammals. 

%
\section*{Methods}
% is mainly teak and mopane woodland, on Kalahari sands, interspersed with open plains along ancient river systems

\subsection*{Data}
The population of lions in Hwange National Park in North-Western Zimbabwe has been studied since 1999. The 5884 \emph{?} km\textsuperscript{2} field site lies in the northern range of the park, where it receives low seasonal rainfall and consists mostly of woodland and scrubland scattered with little open or bushed grassland, that covers in total only one tenth of the park \citep{Loveridge:2007ic,Rogers:1993vs}. HNP boarders on safari areas, hunting concessions, communal land, and wildlife management areas. Concessions buffer areas of the park and of human settlements in the north and north east. Communal land, mainly used for subsistence agriculture and wildlife exploitation under the Communal Areas Management Plan for Indigenous Resources (CAMPFIRE), forms the southern neighbouring area. In the south and south west, the park shares a boarder with wildlife management areas of Botswana.   
The collection of data span the period from 1999 to 2013. During this time, most adult males and at least one female per pride wore radio-collars. Radio telemetry enabled the deliberate localisation of prides, and male nomads and coalitions who were in the field site, at least once per month. Field staff identified individual lions from whisker spot patterns along with other telling marks like scars and teeth characteristics \citep{Becker:2013ib}. Censuses recorded new arrivals and dissapearNewborns and individuals that were born before the start of the study or outside of the field site received at first sighting an individual identifier and a birth date backdated by the estimated age. 

\citep{Mosser:2009fl, Packer:2005dr}.  Since 1984, when radio telemetry first enabled deliberate localisation of all prides, demographic censuses of each pride, recording births, deaths, and migrations of all pride members, generally occurred at least once every two weeks. Individual lions were identified from whisker-spot patterns and natural markings \citep{Packer:1991us}.  Dates of birth were inferred from the characteristic behavior of females around parturition \citep{Packer:2001wn}. Female pride mates often give birth at the same time and communally rear litters \citep{Packer:1988ux}, and when this occurred, the size of the communal litter and the number of mothers were recorded. In total, we had demographic records on 4393 individuals up until 2010. Morphological data were collected  from 42 prides between 1984 and 2009.  As a proxy for body size, we used a lion's breast circumference known as heart girth \citep{Bertram:1975voa}. These measurements were taken whenever a lion was immobilized. This happened on an ad hoc basis for a variety of reasons. For model parametrization, we used a total of 291 female and 203 male heart girth measurements. We had repeat measures at different ages for 52 females and 32 males. We had heart girth measurements of female offspring at different ages for 82 females and of male offspring for 53 females. 
\subsection{Missing times of birth and death in lifespan data}
In CRR studies, missing data on individual times of birth ($b_i$) and death ($d_i$) can arise from several processes. If any process affects one sex stronger than the other, one sex will be overrepresented in the missing data. Sex-specific processes need to be clearly acknowledged when modelling age-specific survival through imputing the unobservable age states. One of these processes is male-biased dispersal. Dispersing males are commonly lost for data collection and form right-censored records, whereas immigrating males have unobserved birth times that result in left-truncated records. Other processes equally affect both sexes. Among them is the left-truncation of individuals that were born before the study started and the right-censoring of individuals that died after the end of the study. The proportion of these individuals in the missing data increases if the life span of individuals is long in comparison to the study period. A CRR data set therefore consists of a mixture of uncensored, left-truncated, and right-censored records.


in our example, the data that are technically collected in a CRR scheme are collapsed/collated into individual first seen and last seen 
Our data different but we use consistent notation to colchero et al and point out the differences. While normally data collection for CRR studies consists of repeated sampling occurring between the start ($t_1$) and end ($t_T$) of the study period at discrete time intervals $[t_1, ..., t_T]$, the re-sighting data recorded from our field site are collapsed into individual records of birth and death or last seen times. Therefore, in our case ($t_1$) and ($t_T$) designate the start and end of the study period

Sex-specific dispersal  in missing data on females  means something else than missing data on males due to male-biased dispersal, systematic sex bias in truncated and censored records with observations on males being almost by default censored or truncated. Because re-sighting probabilities are very high as long as individuals are within the study area, re-sighting probability for this duration does not introduce further uncertainty. 
The data collection for occurred in a study period, spanning an interval of $[t_1, ..., t_T]$, where $t_1$ and $t_T$ correspond to the start and end of the study period, respectively. Instead of marking individuals, researchers use whisker spot patterns along with other telling marks like scars and teeth size and wear\citep{Becker:2013ib}.

\subsection{Model}
\subsection{Model selection}

\subsection{Simulation analysis}
\section{Results}
\subsection{Age-specific survival by sex of the Hwange lions, biological meaning: internal vs. external mortality}
\subsection{Simulation results: sensitivity of age-specific survival estimates to male age-specific resighting probability}
\section{Discussion}
A paragraph on demographic models, population management and conservation.
A paragraph on internal vs. external mortality, the effect of hunting (compensatory or additive?, starting point: Pollock 2000)
A paragraph on applications to other sex-biased dispersal species.	 
% Lions adapt sex ratio at birth to compensate for adult male losses (Becker, Pusey)`
\section{Acknowledgements}